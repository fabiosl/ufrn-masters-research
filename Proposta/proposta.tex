\documentclass{article}
\begin{document}
\title{Proposta de Mestrado}
\author{Fabio Leal}
\maketitle  





\section{Background}
The goal of this chapter is to present the technical concepts for a better understanding of our job. 


\subsection{Service Level Agreements (SLA)}
According to the \textit{ITILv3's} official glossary \cite{itilv3glossary}, a Service Level Agreement (SLA) is \textit{an agreement between an IT service provider and a customer.  A service level agreement describes the IT service, documents service level targets, and specifies the responsibilities of the IT service provider and the customer.} 

An SLA is, therefore, a set of measurable constraints that a service provider must guarantee to its customers. 


\subsection{Cloud Computing}
On the early 90's it was commonplace for every Information Technology (IT) company to have its own Data Center with lots of huge servers and main-frames. IT costs were high, and high-performance computing was available only for big companies, as data centers required a lot of physical space and high costs for maintenance. 

The regular way of building a web application was to use client-server approach, where the server was an extremely powerful (and expensive) machine. However, new companies, such as Google, were rising with bigger missions: \textit{``to organize the world's information and make it universally accessible and useful''}. It was \textit{just} impossible to store the petabytes of daily-generated data in a single server. 

From this point, they also realized that it was way cheaper to build and maintain several low-performance servers than a single high-performance machine. This approach, however, was incompatible with the traditional way of building applications, as they were designed to work with a single server and database. 

Lots of research were conducted in this area and a common solution was rising: to distribute data storage and processing. Google, Yahoo and other big IT players helped to build open source tools to make this approach possible, such as Hadoop.

This revolution brought to life new concepts, such as Infrastructure as a Service \textit{(IAAS)}, Platform as a Service \textit{(PAAS)} and Software as a Service \textit{(SAAS)}. 

\cite{AViewOfCloudComputing} says that \textit{Cloud computing refers to both the applications delivered as services over the internet and the hardware and systems software in the data centers that provide those services.} 


\subsection{Data Integration}


\subsection{Web Services \& SOA}















\section{Bibliographic Review}
\subsection{SLA Guided Data integration on data environments}














\newpage
\bibliographystyle{plain}
\bibliography{fabiosmastersbib}	



\end{document}