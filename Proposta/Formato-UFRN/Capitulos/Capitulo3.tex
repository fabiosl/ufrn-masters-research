% Cap�tulo 3
\chapter{Validation}\label{validationChap}

To validate our guidelines, we searched for open-source tools that could possibly require any sort of database transition. As stated on section xxxxxx, a database migration is only justified when xxxxxxxx. 

This condition is only verifiable on production or simulated environments. Wordpress, world's most used CMS (xxxxxx) is built on the top of a relational database, such as PostgreSQL and MySQL. So, to justify a database transition on a Wordpress-based application, we had to (i) have access to an existing deploy of the CMS or (ii) build a simulated environment and load it with fake posts.

Having access to a production version of a heavily used tool, as Wordpress, is not easy, as the owner of the deployment must trust that the access will only be used for research purposes. 

Other open source tools were considered on this phase of the research, such as Redmine (xxxxx) and Moodle (xxxxx), but the same problem that we had finding a Wordpress production environment can be established for these other tools. 

Besides that, famous open source tools usually have a large user-base. This results in a large support of the community towards the development of the software. With that, if a database transition scenario from RDBMs to NoSQL is needed, it possibly have plugins/addons from the community, as \cite{xxxxxxx} and \cite{xxx}

To simplify the process of validating the Guidelines proposed on the previous chapter, we have built a scenario that could be easily mapped to real-world application. The application proposed on the following section is used on Scenarios 1, 2 and 3.

\section{Building the Environment}

% Para validar o processo mostrado no cap 2, construimos o core de um  schema de banco de dados de uma aplicacao que sirva para classificar posts de redes sociais. 

% Na visao do usuario dessa aplicacao, as features possiveis sao: 
% \begin{enumerate}
% \item{Buscar posts que mencionem conteudo X: }
% \item{Classificar (adicionar tags) a um conjunto de posts: n tags podem ser adicionadas}
% \item{Classificar sentimento de cada post: somente um coisa pode mostrar q sentimento (unico field)}
% \item{Arquivar: (um booleano). Uma feature importante desse nosso cenario eh que o usuario pode querer arquivar todos os posts que contenham uma caracteristica X. Por exemplo: Quero arquivar todos os posts de 21/01/01 a 10/03/2015 que contenham a tag Y. }

% \end{enumerate}

% O schema inicial da aplicacao foi constru�do para ser o mais otimizado poss�vel para ambientes de producao, entao optou-se inicialmente por construir uma �nica tabela, fazendo com que o schema da aplicacao fosse totalmente denormalizado. 


% A figura X mostra quais sao os campos dessa tabela atrav�s de uma representacao do modelo de entidades e relacionamentos.


% \section{SLA da aplicacao}

% Seguindo o modelo de XYZ, foi definido o seguinte SLA para essa aplicacao: 

% \begin{enumerate}
% \item{Toda operacao de search deve seguir um SLA onde 98\% das operacoes devem ser executadas em ate 2s.}
% \item{Recuperacao de posts tagueados com um conjunto de tags devem ser retornados em ate 3s}
% \item{Uma operacao de select all nao deve demorar mais do que 1s para ser executada.}
% \end{enumerate}

% \section{A aplicacao}

% To prepare our database to the scenarios that will be discussed on this section, some tasks were necessary: 

% \begin{enumerate}
% \item{Criar instancia para receber os dados da aplicacao. 
% Uma instancia small da amazon <colocar as especificacoes da instancia aqui> foi instalada nesse servidor da amazon. Dizer tambem que nessa instancia rodava mysql (RDBMS) Colocar tambem todo o software que foi envolvido.}
% \item{\textbf{Retrieve data to be used on the scenarios:} We have developed a script to retrieve posts from Facebook and store them on a relational database (MySQL). More than 3 milion posts were captured and are available on ______. TODO: Upload dump file on Github. Essa instancia eh a primaria e era onde o script jogava os posts coletados a partir do Facebook. Os mesmos eram armazenados em um MySQL (RDBMs). }
% \end{enumerate}

% \section{Scenario 01}



% Uma possivel claim dos usuarios desse sistema � de que a busca estah extremamente lenta. De fato, quando existiam poucos posts, podia-se notar que a busca era executada em um tempo aceitavel (mostrar busca com 500K, 1.5KK e 3KK).

% Nossos guidelines dizem que o primeiro passo para propor uma migracao de BDs eh identificar que um requirement da aplicacao foi quebrado. 
% Nesse caso, o requirement de 

% ``Toda operacao de search deve seguir um SLA onde 98\% das operacoes devem ser executadas em ate 2s''

% Assim, o step 1 foi corretamente identificado. 

% O segundo passo eh implementar um runnable sla que possa mostrar que aquela metrica do SLA realmente nao esta sendo cumprida. Esse SLA checker foi implementado em Python e esta disponivel em exp01_main.py. 

% O que esse script faz eh isso: xxxxxxxxxxx.

% O passo 3 seria mostrar (executar o script e realmente mostrar que o requisito esta quebrado). 

% O quarto passo desse nosso step seria propor modificacoes no banco de dados existente. esse cara: http://blog.scoutapp.com/articles/2014/12/19/from-mysql-full-text-search-to-elasticsearch reporta algumas opcoes para melhorar a performance de um full text search com MySQL. 

% Dentre as opcoes para melhorar o processo de search, ele aponta duas solucoes: (i) ``To support full-text search, we needed to use the MySQL MyISAM storage engine. This has major downsides, the primary one being full table locks: when a table is updated, no other changes to that table can be performed.'' A outra saida era (ii) ``We ended up doing this. It was a fairly simple step and allowed us to switch to the InnoDB engine on the master, eliminating the table lock issues.

% This bought us some time, but it wasn't a long-term solution: we basically were rolling our own search and this frequently involved complex queries that third-party search libraries could perform more efficiently. We ended up with massive queries composed of many JOINs plus AND/ORs - these aren't easy to maintain.

% Besides query complexity, it's tough to beat the performance of a dedicated search solution. Our tables have considerable update activity, so this would result in sometimes-significant performance issues.
% ''
% Outros contra-pontos para implementar fulltext search em MySQL apontado por XXX, XXX and XXX sao X Y e Z. 

% Apache Solr, Amazon CloudSearch and Elasticsearch are Search Engines que sao mais robustas para isso. Logo, pegamos o ES que parece mais facil e bons relatos na industria sobre ele 

% O passo 5 sugere que um novo modelo de dados para essa an�lise pura e simples seria essa: XXX. ES � schemaless, entao podemos colocar a mesma coisa. sendo assim, esse eh um exemplo de documento v�lido no ES de Origem.

% Figura de um Documento v�lido.

% Para fazer o dump dessa coisa foi feito um script, o python-mysql-to-es-importer. A importa�ao atraves desse script era inviavel pq sempre ficava caindo a conexao com o MySQL e pq nao era feita com Bulk insert.https://github.com/fabiosl/ufrn-masters-research/blob/master/exp01/python-mysql-to-es-importer/importer.py

% Usamos entao o https://github.com/jprante/elasticsearch-jdbc importer

% Com isso, conseguimos fazer com que o passo 6 fosse executado com sucesso.

% Um novo runnable SLA era necess�rio para comparar os tempos de execucao do antigo com o novo. Esse runnable SLA esta disponivel em ZZZZ. 

% Comparando os resultados de um e dois, eh possivel dizer que uma migracao ou estrategia multibanco pode ser util para esse caso.



% Mostrar que a velocidade de busca antes era respeitada e algo deveria ser feito. 


% To have a wide variety of 
% - Recupera�ao de dados:
    
% - Colocar 5 linhas de exemplo 

% - Instala�ao de servidores
%     - Na minha conta da AWS criei duas inst�ncias para o experimento 1. Tive um retrabalho aqui, pois as primeiras inst�ncias que lancei (T1.Micro) nao aguentaram a carga dos 3 milh�es.
%     - Substitu� as duas inst�ncias micro por duas small.
%     - Tive que estudar um pouco sobre os Security Groups da Amazon e tamb�m aprendi a configurar Elastic IPs para ter ips fixos p�s-restart nas minhas m�quinas.

% Criei esse script Python para importar os dados do MySQL (base prim�ria) para o Elasticsearch (2a base): https://github.com/fabiosl/ufrn-masters-research/commit/d18f8582c21d5376b0aa7b9bc394129a67db8662

% \section{Scenario 01}
% Scenario 01 was a Full-Text Search performed on Elasticsearch engine vs MySQL DB.

% Elasticsearch wins.

% \section{Scenario 02}
% Scenario 02 was searching for an unstructured  document (a tag defined by a user, for example). Shows that even with joins is a costy operation and filling it in a text drives us to a Fulltext-search scenario. 

% Elasticsearch wins.
%  % Teste para abreviatura 

% \abrv[UFRN -- Universidade Federal do Rio Grande do Norte]{UFRN}

% \section{Scenario 03}
% Update by query Scenario. 

% Elasticsearch doesn't support natively. 

% Update by query plugin

% MySQL wins
