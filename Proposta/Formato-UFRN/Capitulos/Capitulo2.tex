% Cap�tulo 2
\chapter{Bibliographic Review}\label{bibreview}

To provide us a better understanding over the use of SLAs in component / service migrations, we have performed a Systematic Mapping study to assess the use of SLAs in database transition scenarios, specifically on migrations from relational databases with NoSQL ones. 

This study is available on Appendix \ref{appendix}. 

\section{Identified problems}

As a result, we have analyzed over 70 publications closely related to the use of SLAs in migration scenarios. The study revealed a number of interesting outcomes, and we emphasize two of them below:

\begin{itemize}
\item{No publication was found addressing the problem of measuring the overall improvements after a database transition. Several benchmarking frameworks, such as TPC-H, TPC-DS and YCSB were identified\cite{6616442} during our survey, though. These benchmarking frameworks could be a good starting point to develop new tools and specialized frameworks to solve this problem and might be used in our study to validate that a migration was successful.
}

\item{ \cite{6253526}, \cite{6461875}, \cite{6511780} and \cite{Xiong:2011:APA:2038916.2038931} propose SLA-centric/User-Centric solutions to monitor the performance of web applications. All these solutions are technology-agnostic and could be used to monitor the performance improvements promised by a database transitioning process. Industry experts also pointed out that there are some services, such as New Relic~\cite{newrelic}, Appsee~\cite{appsee} and Datadog~\cite{datadog} that provide SLA-monitoring tools for web apps. 

The systematic mapping revealed no open source solution to monitor Application SLAs in a user-centered view (application level).  
}

\end{itemize}

