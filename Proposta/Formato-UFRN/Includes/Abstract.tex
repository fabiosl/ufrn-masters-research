% Resumo em língua estrangeira (em inglês Abstract, em espanhol Resumen, em francês Résumé)
\begin{center}
	{\Large{\textbf{SLA-Based Guidelines for Database Transitioning}}}
\end{center}

\vspace{1cm}

\begin{flushright}
	Author: Fabio de Sousa Leal\\
	Supervisor: Martin A. Musicante
\end{flushright}

\vspace{1cm}

\begin{center}
	\Large{\textsc{\textbf{Abstract}}}
\end{center}

\noindent Component-based Software Engineering (CBSE) and Service-Oriented Architecture (SOA) became  popular ways to develop software over the last years. During the life-cycle of a software system, several components and services can be developed, evolved and replaced. In production environments, the replacement of core components, such as databases, is often a risky and delicate operation, where several factors and stakeholders should be considered.

Service Level Agreement (SLA), according to ITILv3's official glossary, is ``an agreement between an IT service provider and a customer. The agreement consists on a set of measurable constraints that a service provider must guarantee to its customers.''. In practical terms, SLA is a document that a service provider delivers to its consumers with minimum quality of service (QoS) metrics.

This work is intended to assesses and improve the use of SLAs to guide the transitioning process of databases on production environments. In particular, in this work we propose SLA-Based Guidelines/Process to support migrations from a relational database management system (RDBMS) to a NoSQL one. Our study is validated by case studies.

\noindent\textit{Keywords}: SLA, Database, Migration.



\begin{center}
	\Large{\textsc{\textbf{Resumo}}}
\end{center}

Engenharia de Software Baseada em Componentes (CBSE) e Arquitetura Orientada a Servi\c{c}os (SOA) tornaram-se formas populares de se desenvolver software nos \'ultimos anos. Durante o ciclo de vida de um software, v\'arios componentes e servi\c{c}os podem ser desenvolvidos, evolu\'idos e substitu\'idos. Em ambientes de produ\c{c}\~ao, a substitui\c{c}\~ao de componentes essenciais - como os que envolvem bancos de dados - \'e uma opera\c{c}\~ao delicada, onde v\'arias restri\c{c}\~oes e stakeholders devem ser considerados.

Service-Level agreement (acordo de n\'ivel de servi\c{c}o - SLA), de acordo com o gloss\'ario oficial da ITIL v3 , \'e ``um acordo entre um provedor de servi\c{c}o de TI e um cliente. O acordo consiste em um conjunto de 
restri\c{c}\~oes mensur\'aveis que um prestador de servi\c{c}os deve garantir aos seus clientes.''. Em termos pr\'aticos, um SLA \'e um documento que um prestador de servi\c{c}o oferece aos seus consumidores garantindo n\'iveis m\'inimos de qualidade de servi\c{c}o (QoS).

Este trabalho busca avaliar a utiliza\c{c}\~ao de SLAs para guiar o processo de transi\c{c}\~ao de bancos de dados em ambientes de produ\c{c}\~ao. Em particular, propomos um conjunto de \textit{guidelines} baseados em SLAs para apoiar decis\~oes migra\c{c}\~oes de bancos de dados relacionais (RDBMS) para bancos NoSQL. Nosso trabalho \'e validado por estudos de caso.

\noindent\textit{Keywords}: SLA, Database, Migration.

