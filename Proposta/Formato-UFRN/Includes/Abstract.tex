% Resumo em l�ngua estrangeira (em ingl�s Abstract, em espanhol Resumen, em franc�s R�sum�)
\begin{center}
	{\Large{\textbf{TODO::CHANGE::Guidelines for database transitioning on production environments}}}
\end{center}

\vspace{1cm}

\begin{flushright}
	Author: Fabio de Sousa Leal\\
	Supervisor: Martin A. Musicante
\end{flushright}

\vspace{1cm}

\begin{center}
	\Large{\textsc{\textbf{Abstract}}}
\end{center}

\noindent Component-based Software Engineering (CBSE) and Service-Oriented Architecture (SOA) became  popular ways to develop software over the last years. During the life-cycle of a software system, several components and services can be developed, evolved and replaced. In production environments, the replacement of core components, such as databases, is often a risky and delicate operation, where several factors and stakeholders take place.

Service Level Agreements (SLA), according to ITILv3's official glossary, is ``an agreement between an IT service provider and a customer. The agreement consists on a set of measurable constraints that a service provider must guarantee to its customers.''. In practical terms, it is a document that a service provider delivers to its consumers with minimum quality of service (QoS) metrics.

This work assesses and improves the use of SLAs to guide the transitioning process of databases on production environments. In particular, in this work we propose SLA-Based Guidelines/Process to support migrations from a relational database management system (RDBMS) to NoSQL one. Our study is validated by case studies.

\noindent\textit{Keywords}: SLA, Database, Migration.