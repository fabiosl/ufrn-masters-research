% Resumo em l�ngua estrangeira (em ingl�s Abstract, em espanhol Resumen, em franc�s R�sum�)
\begin{center}
	{\Large{\textbf{Guidelines for database transitioning on production environments}}}
\end{center}

\vspace{1cm}

\begin{flushright}
	Author: Fabio de Sousa Leal\\
	Supervisor: Martin A. Musicante - Associate Professor
\end{flushright}

\vspace{1cm}

\begin{center}
	\Large{\textsc{\textbf{Abstract}}}
\end{center}

\noindent Component-based Software Engineering (CBSE) and Service-Oriented Architecture (SOA) became  popular ways to develop software over the last years. During the life-cycle of a software, several components and services can be developed, evolved and replaced. In production environments, the replacement of a component or service is often a risky and delicate operation, where several factors and stakeholders take place.

Service Level Agreements (SLA), according to ITILv3's official glossary, is ``an agreement between an IT service provider and a customer. The agreement consists on a set of measurable constraints that a service provider must guarantee to its customers.''. In practical terms, it is a document that a service provider delivers to its consumers with minimum quality of service (QoS) metrics.

This work aims to assess and improve the use of SLAs to guide transition/replacement scenarios in software components, such as databases, business logic and user interface. Our study will focus on database transition scenarios, where we want to assess the usage of SLA-Guided processes to support migrations from a relational database management system (RDBMS) to NoSQL. Our study will be validated by case studies.

\noindent\textit{Keywords}: SLA, Database, Migration.